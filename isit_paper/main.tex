\documentclass[conference]{IEEEtran}
\pdfoutput=1

\usepackage[sorting=none,style=ieee,backend=bibtex]{biblatex}
\addbibresource{references.bib}

\usepackage{amsmath}
\usepackage{amsbsy}
\usepackage[noend]{algpseudocode}
\usepackage{graphicx}
\usepackage{mathtools}
\usepackage{amssymb}
\usepackage{eqnarray}
\usepackage{multicol}
\usepackage{subcaption}
\usepackage{listings}
\usepackage{color}
\usepackage{units}
\usepackage{amsthm}
\usepackage{amsmath}
\usepackage{vwcol}
\usepackage{caption}
\usepackage{hyperref}
\usepackage{footnote}
\usepackage{blkarray}
\usepackage{comment}

\newtheorem{theorem}{Theorem}
\newtheorem{remark}{Remark}
\newtheorem{corollary}{Corollary}[theorem]
\newtheorem{lemma}[theorem]{Lemma}
\newtheorem{example}{Example}[section]
\captionsetup[table]{position=below}
\def\tablename{Table}
\makesavenoteenv{tabular}
\makesavenoteenv{table}

\allowdisplaybreaks

\ifCLASSINFOpdf

%%%%%%%%%%%%%%%%%%%%%%
\newcommand{\mbf}[1]{\ensuremath{\mathbf{#1}}}
\newcommand{\bs}[1]{\ensuremath{\boldsymbol{#1}}}
\newcommand{\mc}[1]{\ensuremath{\mathcal{#1}}}
\newcommand{\cxy}{\ensuremath{C_{X\rightarrow Y}}}
\newcommand{\cyx}{\ensuremath{C_{Y\rightarrow X}}}
\newcommand{\estcxy}{\ensuremath{\hat{C}_{X\rightarrow Y}}}
\newcommand{\estcyx}{\ensuremath{\hat{C}_{Y\rightarrow X}}}
\newcommand{\optcxy}{\ensuremath{C^*_{X\rightarrow Y}}}
\newcommand{\optcyx}{\ensuremath{C^*_{Y\rightarrow X}}}
\newcommand{\fxr}[1]{\ensuremath{f_{X_#1}^{(r)}}}
\newcommand{\fxc}[1]{\ensuremath{f_{X_#1}^{(c)}}}
\newcommand{\fyr}[1]{\ensuremath{f_{Y_#1}^{(r)}}}
\newcommand{\fyc}[1]{\ensuremath{f_{Y_#1}^{(c)}}}
\newcommand{\estfxr}[1]{\ensuremath{\hat{f}_{X_#1}^{(r)}}}
\newcommand{\estfxc}[1]{\ensuremath{\hat{f}_{X_#1}^{(c)}}}
\newcommand{\estfyr}[1]{\ensuremath{\hat{f}_{Y_#1}^{(r)}}}
\newcommand{\estfyc}[1]{\ensuremath{\hat{f}_{Y_#1}^{(c)}}}
\newcommand{\optfxr}[1]{\ensuremath{f_{X_#1}^{(r)*}}}
\newcommand{\optfxc}[1]{\ensuremath{f_{X_#1}^{(c)*}}}
\newcommand{\optfyr}[1]{\ensuremath{f_{Y_#1}^{(r)*}}}
\newcommand{\optfyc}[1]{\ensuremath{f_{Y_#1}^{(c)*}}}
\newcommand{\history}[1]{\ensuremath{\mc{H}_#1}}
\newcommand{\hr}[1]{\ensuremath{\history{i}^{(r)}}}
\newcommand{\hc}[1]{\ensuremath{\history{i}^{(c)}}}
\newcommand{\kl}[2]{\ensuremath{D(#1 \mid \mid #2)}}
\newcommand{\argmin}[1]{\ensuremath{\underset{#1}{\operatorname{argmin}}}}
\newcommand{\argmax}[1]{\ensuremath{\underset{#1}{\operatorname{argmax}}}}
\newcommand{\refclass}{\ensuremath{\tilde{\mc{P}}}}
\newcommand{\refclassr}{\ensuremath{\refclass^{(r)}}}
\newcommand{\refclassc}{\ensuremath{\refclass^{(c)}}}
\newcommand{\mcal}[1]{\ensuremath{\mathcal{#1}}}

%%%%%%%%%%%%%%%%%%%%%%


\begin{document}

\title{A Sample Path Measure of Causal Influence}
%

\author{
Gabriel~Schamberg, \emph{Student Member, IEEE},
Todd~P.~Coleman, \emph{Senior Member,~IEEE}
}

\maketitle

%%%%%%%%%%%%%%%%%%%%%%%%%%%%%%%%%%%%%%%%%%%%%%%%%%%%%%%%%%%%%%%%%%%%%%%%%%%%%%%

\begin{abstract}
We present a sample path dependent measure of causal influence between two time series. The proposed measure is a random variable whose expected value is the directed information rate when such a rate exists. A realization of the proposed measure may be used to identify the specific patterns in the data that yield a greater flow of information from one process to another. We demonstrate how sequential prediction theory may be leveraged to obtain accurate estimates of the causal measure at each point in time and introduce a notion of regret for assessing the performance of estimators of the measure. We prove a finite sample bound on this regret that is determined by the regret of the sequential predictors used in obtaining the estimate. We estimate the causal measure for a simulated collection of binary Markov processes using context tree weighting with side information. Finally, given that the measure is a function of time, we demonstrate how estimators of the causal measure may be extended to function in time-varying scenarios.
\end{abstract}

%%%%%%%%%%%%%%%%%%%%%%%%%%%%%%%%%%%%%%%%%%%%%%%%%%%%%%%%%%%%%%%%%%%%%%%%%%%%%%%

% Note that keywords are not normally used for peerreview papers.
\begin{IEEEkeywords}
Granger Causality, KL-Divergence, Sequential Prediction, Markov Chains.
\end{IEEEkeywords}

\IEEEpeerreviewmaketitle

%%%%%%%%%%%%%%%%%%%%%%%%%%%%%%%%%%%%%%%%%%%%%%%%%%%%%%%%%%%%%%%%%%%%%%%%%%%%%%%

\section{Introduction}

\textcolor{red}
{\begin{itemize}
    \item Granger Causality
    \item Directed Information
    \item Sequential Prediction
    \item Argument (neuroscience?) for why sample path causality is important
\end{itemize}}


\section{Sample Path Measure of Causal Influence}

\begin{comment}
- Definition
- True causal measure is random
- Can be applied to time varying settings
- Always non-negative
- Is always well defined (as opposed to directed information rate)
- ``Semi-local'' - Measure such as DI and TE are functions of the model (can't capture local influence), but local measures like pointwise mutual information can be negative. Our measure is local and positive.
\end{comment}

We begin by defining arbitrary measurable spaces \mcal{X}, \mcal{Y}, and \mcal{Z}. Suppose we observe the stochastic processes $X^n \in \mcal{X}^n$, $Y^n \in \mcal{Y}^n$, and $Z^n \in \mcal{Z}^n$, characterized by the joint probability density function (pdf) $f_{X^n,Y^n,Z^n}(x^n,y^n,z^n)$. We begin by considering the scenario where, having observed $(x^{i-1},y^{i-1},z^{i-1})$, we wish to determine the causal influence that $y^{i-1}$ has the next observation $x^i$. In such a scenario, we consider the following \emph{restricted} (denoted $(r)$) and \emph{complete} (denoted $(c)$) conditional densities:

\begin{eqnarray}
\fxr{i}(x_i) \triangleq f_{X_i \mid X^{i-1},Z^{i-1}}
    (x_i \mid x^{i-1},z^{i-1}) \\
\fxc{i}(x_i) \triangleq f_{X_i \mid X^{i-1},Y^{i-1},Z^{i-1}}
    (x_i \mid x^{i-1},y^{i-1},z^{i-1}).
\end{eqnarray}

\noindent Using these densities, at each time $i$ we define the sample path measure of causality from $Y^n$ to $X^n$ for a given realizations $(x^{i-1},y^{i-1},z^{i-1})$ as:

\begin{equation}
\cyx(x^{i-1},y^{i-1},z^{i-1}) = \kl{\fxc{i}}{\fxr{i}}.
\end{equation}

\noindent For ease of notation, we may represent the causal measure at time $i$ simply as $\cyx(i)$.

The key observation that must be made that \fxc{i} and \fxr{i} are determined by the realizations of $X^n$, $Y^n$, and $Z^n$. As a result, \emph{the causal measure is a random variable}. In this regard, our causal measure is different from previous measures of causality wherein the causal influence is determined by the model, and not the sample path. To ensure this point is made clear, we will present an example.

\begin{example}
Suppose $Y_i \sim \text{Bern}(0.2)$ iid for $i=1,2,\dots$ and:

\begin{equation}
X_i \sim
\begin{cases}
      \text{Bern}(0.9), & Y_{i-1} = 1 \\
      \text{Bern}(0.5), & Y_{i-1} = 0
\end{cases}
\end{equation}

\noindent Intuitively, we would expect that in some sense $Y^n$ is ``causing'' $X^n$ to a greater extent when $Y_i$ is one than when it is zero. In order formalize this, we have to find the probability of $X_i=1$ when only $X^{i-1}$ is known:

\begin{equation*}
\begin{aligned}
\mathbb{P}(&X_i = 1 | X^{i-1}=x^{i-1}) \\
&= \mathbb{P}(X_i = 1) \\
&= \sum_{y_{i-1}\in \{0,1\}}
    \mathbb{P}(X_i =1 \mid Y_{i-1} = y_{i-1}) \mathbb{P}(Y_{i-1} = y_{i-1}) \\
&= (0.5)(0.8) + (0.9)(0.2) \\
&= 0.58.
\end{aligned}
\end{equation*}

\noindent We can fully characterize the complete and restricted probability mass functions (pmfs) using these probabilities, i.e. $\fxr{i}(1) = 0.58$, $\fxc{i}(1) = 0.9$ if $y_{n-1}=1$, and $\fxc{i}(1) = 0.5$ if $y_{n-1}=0$. We can now compute the causal measure, which takes on one of two values determined by the observation $y_{i-1}$:

\begin{equation}
\cyx(i) =
\begin{cases}
      0.363, & y_{i-1} = 1 \\
      0.019, & y_{i-1} = 0
\end{cases}
\end{equation}

\noindent Thus, we see that our measure captures how, even in a stationary Markov chain, different patterns in the observed data may give rise to different levels of causal influence. By contrast, we note that because the process is stationary, the directed information rate and transfer entropy are both given simply by $E[\cyx]=(0.9)(0.019)+(0.1)(0.363)=0.088$.
\end{example}

The above example gives rise to two key observations. First, transfer entropy and directed information rate fail to capture that the causal effect of $\{Y\}$ upon $\{X\}$ varies in time, even in the simplest of stationary processes. Second, by averaging over all possible histories, transfer entropy and directed information rate are minimally affected by patterns that occur with low probability, even if those patterns induce a high level of causal influence.

We now discuss some key properties of the proposed causal measure. First, we note the crucially important quality of non-negativity, which follows directly from the fact that the measure is nothing more than a KL-Divergence. Next, we characterize our measure as being ``semi-local.'' This term is meant to illustrate that our measure is in a sense between existing measures or causality. In particular, we note that the directed information (rate), Granger causality, transfer entropy, etc. are all \emph{expectations}, determined entirely by the underlying probabilistic model of the observed data. On the other end of the spectrum, local data-dependent versions of these measures may be obtained by substituting the self information for entropy. Unfortunately, these local versions of causal measures may negative when unlikely sequences occur. Our measure can be thought of as being determined by the underlying statistical model, but the parameters of the model are determined by the past data. We use the term ``semi-local'' because at any given time, the measure is determined by the observations from the past and an expectation over the future.





\begin{comment}
\noindent This process can be equivalently characterized by the four-state ``complete'' Markov Chain with states $(X,Y) \in \{0,1\}^2$ and transition matrix:

\begin{equation}
M^{(c)} =
\begin{blockarray}{ccccc}
(0,0) & (0,1) & (1,0) & (1,1) \\
\begin{block}{(cccc)c}
    0.25 & 0.1 & 0.25 & 0.1 & (0,0) \\
    0.25 & 0.9 & 0.25 & 0.9 & (0,1) \\
    0.25 & 0.1 & 0.25 & 0.1 & (1,0) \\
    0.25 & 0.9 & 0.25 & 0.9 & (1,1) \\
\end{block}
\end{blockarray}
\end{equation}

\noindent where $M^{(c)}_{kj}$ represents the probability of having $(X_i,Y_i)$ in state $k$ given that $(X_{i-1},Y_{i-1})$ is in state $j$. We can additionally define a two-state ``restricted'' Markov Chain with states $X \in \{0,1\}$ and transition matrix:

\begin{equation}
M^{(r)} =
\begin{blockarray}{ccc}
0 & 1 \\
\begin{block}{(cc)c}
    0.3 & 0.3 & 0 \\
    0.7 & 0.7 & 1 \\
\end{block}
\end{blockarray}
\end{equation}
\end{comment}
\section{Estimation of the Causal Measure}

\begin{comment}
- Sequential Prediction
- Notion of causality regret - whereas DI has one value that the Tsachy paper tries to converge to, we are looking more at a sequential prediction problem because we want to estimate causality accurately at every point in time
- Theorem
- Independent selection of restricted and complete reference classes - addresses problem with restricted distribution being infinite order raised in Purdon paper
- Discussion of estimating with time varying statistics
- Can be computed online
\end{comment}

Assuming the joint statistics underlying the observed data are unknown, it is necessary to develop methods for estimating the causal measure. As such, an estimate of the causal measure can be obtained by simply estimating the complete and restricted distributions and then computing the KL divergence between the two at each time. Such an estimator allows us to leverage results from the field of sequential prediction.

The sequential prediction problem formulation we consider is as follows: A learner is sequentially observing a sequence $x_1,x_2,\dots,x_n$ over some space of observations $\mc{X}$. At each round, $i$, having observed the sequence $x_1,\dots,x_{i-1}$ (or more generally, the history $\history{i}$, which may include information in addition to $x^{i-1}$), the learner selects a probability assignment $\hat{f}_i \in \mc{P}$, where $\mc{P}$ is the space of probability distributions over $\mc{X}$. Once $\hat{f}_i$ is chosen, $x_i$ is revealed and a loss $l(\hat{f}_i,x_i)$ is incurred by the learner, where the loss function $l:\mc{X}\rightarrow \mathbb{R}$ is chosen to be the self-information loss given by:

\begin{equation}
l(f,x) = -\log(f(x))
\end{equation}

The performance of sequential predictors is typically assessed using a notion of \emph{regret} with respect to a reference class of probability distributions $\refclass \subset \mc{P}$. For a given round $i$ and reference distribution $\tilde{f}_i \in \refclass$, the learner's regret is given by the difference in loss of the chosen probability assignment and the loss of the reference distribution:

\begin{equation}
r(\tilde{f}_i,x_i) = l(\hat{f}_i,x_i) - l(\tilde{f}_i,x_i)
\end{equation}

\noindent In many cases the performance of sequential predictors will be measured by the worst case regret, given by:

\begin{align}
R_n(\refclass_n) &= \sup_{x^n \in \mc{X}^n} \sum_{i=1}^n l(\hat{f}_i,x_i) - \inf_{\tilde{f}\in \refclass_n} \sum_{i=1}^n l(\tilde{f}_i,x_i) \label{optimal_f} \\
&\triangleq \sup_{x^n \in \mc{X}^n} \sum_{i=1}^n r(f^*_i,x_i)
\end{align}

\noindent where $f^*_i \in \refclass$ is defined as the distribution from the reference class with the smallest cumulative loss up to time $n$. We also define $f^* \in \refclass_n \subset \mc{P}^n$ to be the cumulative loss minimizing \emph{joint} distribution, noting that the reference class of joint distributions $\refclass_n$ is not necessarily equal to $\refclass^n$ (i.e. $\refclass \times \refclass \times \dots$), as often times there may be a constraint on the selection of the best reference distribution that is imposed in order to establish bounds. In the absence of any restrictions, the reference distributions may be selected at each time such that $f^*_i(x_i)=1$, resulting in zero cumulative loss for any sequence $x^n$. Thus, bounds on regret often assume stationarity by enforcing $f_1^*=f_2^*=\dots=f_n^*$ or assume that $f_i^* = f^*_{i+1}$ for all but some small number of indices. For various learning algorithms (i.e. strategies for selecting $\hat{f}_i$ given $\history{i}$) and reference classes $\refclass_n$, these bounds on the worst case regret are established as a function of the sequence length $n$:

\begin{equation}
R_n(\refclass_n) \le M(n)
\end{equation}

It follows naturally that an estimator for our causal measure can be constructed by building two sequential predictors. The restricted predictor $\estfxr{i}$ computed at each round using $\hr{i} \triangleq \{x_1,\dots,x_{i-1}\} \cup \{z_1,\dots,z_{i-1}\}$, and the complete predictor $\estfxc{i}$ computed at each round using $\hc{i} \triangleq \{x_1,\dots,x_{i-1}\} \cup \{y_1,\dots,y_{i-1}\} \cup \{z_1,\dots,z_{i-1}\}$. It then follows that each of these predictors will have an associated worst case regret, given by $R^{(r)}_n(\refclassr_n)$ and $R^{(c)}_n(\refclassc_n)$, where $\refclassr_n$ and $\refclassc_n$ represent the restricted and complete reference classes. Additionally we will define the instantaneous regrets for each of our predictors as $r^{(r)}$ and $r^{(c)}$. Using these sequential predictors, we define our estimated causal influence from $Y$ to $X$ at time $n$ as:

\begin{equation}
\estcyx(i) = \kl{\estfxc{i}}{\estfxr{i}}
\end{equation}

To assess the performance of an estimate of the causal measure, we define a notion of causality regret:

\begin{equation}
CR(n) \triangleq \sum_{i=1}^n \left| \estcyx(i) - \optcyx(i)  \right|
\end{equation}

\noindent where we define:

\begin{equation}
\optcyx(i) = \kl{\optfxc{i}}{\optfxr{i}}
\end{equation}

\noindent with $\optfxc{i} \in \refclass^{(c)}$ and $\optfxr{i} \in \refclass^{(r)}$ defined as the loss minimizing distributions from the complete and restricted reference classes. We note that with this notion of causal regret, the estimated causal measure is being compared against the best estimate of the causal measure from within a reference class. As such, we limit our consideration to the scenario in which the reference classes are sufficiently representative of the true sequences to produce a desirable $\optcyx$ (i.e. $\optcyx(i) \approx \cyx(i)$ for all $i$).

We now present the necessary preliminaries for proving a finite sample bound on the estimates of causality regret for the special case when $\mc{X}$ is a discrete space. We begin by introducing an assumption that requires the underlying sequence and the complete and restricted sequential predictors to be well-behaved in an appropriate sense.

\begin{assumption} \label{assumption:gbound}
For sequential predictors \estfxc{i} and \estfxr{i}, we assume that the collection of observations is such that:
\begin{equation}
\sup_{x \in \mc{X}} \log \frac{\estfxc{i}(x)}{\estfxr{i}(x)} < L \ \ \forall i=1,\dots,n
\end{equation}
\end{assumption}

We now show that the cumulative KL divergence from the best reference distribution to the predicted distribution is less than the predictor's worst-case regret.

\begin{lemma}\label{lemma:kl}
For a sequential predictor $\hat{f}_i$ with worst case regret $M(n)$, a collection observations $(x^n,y^n,z^n)$, and any distribution from the reference class $f \in \refclass_n$:

\begin{equation}
\sum_{i=1}^n \kl{f_i}{\hat{f}_i}\le M(n)
\end{equation}
\end{lemma}

\begin{proof}
\begin{align}
\sum_{i=1}^n \kl{f_i}{\hat{f}_i}
&= \sum_{i=1}^n \sum_{x\in\mc{X}} f_i(x)
    \log \frac{f_i(x)}{\hat{f}_i(x)} \\
&\le \sum_{i=1}^n
    \left[ \sup_{x\in\mc{X}}
    \log \frac{f_i(x)}{\hat{f}_i(x)} \right]
    \sum_{x\in\mc{X}} f_i(x) \\
&= \sum_{i=1}^n \sup_{x\in\mc{X}} r(f_i,x) \\
&\le \sup_{x^n\in\mc{X}^n} \sum_{i=1}^n r(f_i,x_i)\\
&\le \sup_{x^n\in\mc{X}^n} \sup_{f\in\refclass_n} \sum_{i=1}^n r(f_i,x_i)\\
&\le M(n)
\end{align}
\end{proof}

Next, we bound the cumulative difference in expectation of a bounded function between the best reference distribution and sequential predictor.

\begin{lemma}
For a sequential predictor $\hat{f}_i$ with worst case regret $M(n)\ge 1$, a collection observations $(x^n,y^n,z^n)$, cumulative loss minimizing distribution $f^*_i$, and bounded functions $g_i:\mc{X}\rightarrow [-K,K]$ with $K\in\mathbb{R}$:

\begin{equation}
\sum_{i=1}^n \left| E_{f^*_i}[g_i(X)] -
    E_{\hat{f}_i}[g_i(X)] \right| \le
    \frac{\left|\mc{X}\right|K}{\sqrt{2}}\sqrt{n \cdot M(n)}
\end{equation}
\end{lemma}

\begin{proof}
\begin{align}
\sum_{i=1}^n &\left| E_{f^*_i} [g_i(X)] -
    E_{\hat{f}_i}[g_i(X)] \right| \\
&= \sum_{i=1}^n \left| \sum_{x\in\mc{X}}
    \left[ f^*_i(x) - \hat{f}_i(x) \right] g_i(x) \right| \\
&\le \sum_{i=1}^n \sum_{x\in\mc{X}} \left| f^*_i(x) -
    \hat{f}_i(x) \right| \left| g_i(x) \right|
    \label{ref_triangle_eq}\\
&\le \sum_{i=1}^n \sum_{x\in\mc{X}}
    L\sqrt{\frac{1}{2}\kl{f^*_i}{\hat{f}_i}}
    \label{ref_pinsker_assumption}\\
&= \frac{\left|\mc{X}\right| K}{\sqrt{2}} \sum_{i=1}^n
    \sqrt{\kl{f^*_i}{\hat{f}_i}}
\end{align}

\noindent where \eqref{ref_triangle_eq} uses the triangle inequality and \eqref{ref_pinsker_assumption} uses Pinsker's inequality and the boundedness of $g_i$. Focusing now on the sum, we define $\vec{v}$ such that $\vec{v}_i = \sqrt{\kl{f^*_i}{\hat{f}_i}}$ for $i=1,\dots,n$:

\begin{align}
\sum_{i=1}^n \sqrt{\kl{f^*_i}{\hat{f}_i}}
&= \left|\left|\vec{v}\right|\right|_1 \\
&\le \sqrt{n}\left|\left|\vec{v}\right|\right|_2 \label{holders}\\
&= \sqrt{n}\left( \sum_{i=1}^n \kl{f^*_i}{\hat{f}_i}
    \right)^{\frac{1}{2}} \\
&\le \sqrt{n \cdot M(n)} \label{ref_kl_lemma}
\end{align}

\noindent where \eqref{holders} uses H\"{o}lders inequality and \eqref{ref_kl_lemma} uses Lemma \ref{lemma:kl} and the assumption that $M(n) \ge 1$.
\end{proof}

Finally, we can utilize the assumption and lemmas to bound the cumulative causality regret:

\begin{theorem}
Let the worst case regret for the predictors $\estfxr{i}$ and $\estfxc{i}$ be bounded by $R^{(r)}_n(\refclassr_n) \le M^{(r)}(n)$ and $R^{(c)}_n(\refclassc_n) \le M^{(c)}(n)$, respectively. Then, for any collection of observations $(x^n,y^n,z^n)\in \mc{X}^n \times \mc{Y}^n \times \mc{Z}^n$ such that Assumption \ref{assumption:gbound} holds with bound $L$, we have:

\begin{equation} \label{causal_bound}
\begin{aligned}
\sum_{i=1}^n & \left| \estcyx(n) - \optcyx(n) \right| \le \\
&M^{(r)}(n) + M^{(c)}(n) +
    \frac{\left|\mc{X}\right|L}{\sqrt{2}}\sqrt{n \cdot M^{(c)}(n)}
\end{aligned}
\end{equation}
\end{theorem}

\begin{proof}
\textcolor{red}
{Proof is all done except for showing:
\begin{equation}
\sum_{i=1}^n \left| E_{\optfxc{i}} \left[
\log \frac{\optfxr{i}(X)}{\estfxr{i}(X)} \right] \right| \le M^{(r)}(n)
\end{equation}
\noindent which we can see is basically lemma 1 only it's not quite a KL divergence... We can easily say that that non-absolute value is less than the regret, but the only issue is if \optfxc{i} is closer to \estfxr{i} than it is to \optfxr{i}, then we don't know how large of a negative value the information density would take... I think it should be possible to show, just gotta think a little more.
\emph{Update}:
Note that:
\begin{equation}
E_{\optfxc{i}} \left[
\log \frac{\optfxr{i}(X)}{\estfxr{i}(X)} \right] =
\kl{\optfxc{i}}{\estfxr{i}} - \kl{\optfxc{i}}{\optfxr{i}}
\end{equation}
Thus, if we can show that:
$$\kl{\optfxc{i}}{\optfxr{i}} \le \kl{\optfxc{i}}{\estfxr{i}}$$
then we get that $E_{\optfxc{i}} \left[\log \frac{\optfxr{i}(X)}{\estfxr{i}(X)} \right] \ge 0$ and:
\begin{equation}
\begin{aligned}
\sum_{i=1}^n \left| E_{\optfxc{i}} \left[
\log \frac{\optfxr{i}(X)}{\estfxr{i}(X)} \right] \right|
&= \sum_{i=1}^n E_{\optfxc{i}} \left[
\log \frac{\optfxr{i}(X)}{\estfxr{i}(X)} \right] \\
&= \sum_{i=1}^n  \sum_{x\in\mc{X}} \optfxc{i}(x)
\log \frac{\optfxr{i}(x)}{\estfxr{i}(x)} \\
&\le  \sum_{i=1}^n \sup_{x\in\mc{X}} \log \frac{\optfxr{i}(x)}{\estfxr{i}(x)} \sum_{x'\in\mc{X}} \optfxc{i}(x') \\
&= \sum_{i=1}^n \sup_{x\in\mc{X}} \log \frac{\optfxr{i}(x)}{\estfxr{i}(x)} \\
&\le \sup_{x^n\in\mc{X}^n} \sum_{i=1}^n \log \frac{\optfxr{i}(x_i)}{\estfxr{i}(x_i)} \\
&\le M^{(r)}(n)
\end{aligned}
\end{equation}
}
\end{proof}
\section{Simulations}

\textcolor{red}
{\begin{itemize}
    \item Application to conditional bernoulli model using CTW algorithm (Fig 1)
    \item Explicit derivation of bounds and computation of regret
    \item Use changing environment meta algorithm for changing parameters to demonstrate that it appears to work even though theorem doesn't apply (Fig 2)
\end{itemize}}

We begin by demonstrating the estimation of the proposed causal measure on a collection of jointly $k^{th}$-order Markov binary processes $\{X\}$, $\{Y\}$, and $\{Z\}$. In particular, for each of the three processes, we fully characterize the conditional pmf by the probability of observing a one given the most recent $k$ samples of all three processes:

\begin{equation}
p^{(c)}_{X_i},p^{(c)}_{Y_i},p^{(c)}_{Z_i} : \mcal{X}^k \times \mcal{Y}^k \times \mcal{Z}^k \rightarrow [0,1]
\end{equation}

\noindent where

\begin{equation*}
\begin{aligned}
p^{(c)}_{X_i}&(x_{i-k}^{i-1},y_{i-k}^{i-1},z_{i-k}^{i-1}) \\
&\triangleq \mathbb{P}(X_i = 1 \mid X_{i-k}^{i-1} = x_{i-k}^{i-1},Y_{i-k}^{i-1} = y_{i-k}^{i-1},Z_{i-k}^{i-1} = z_{i-k}^{i-1})
\end{aligned}
\end{equation*}

\noindent with $p^{(c)}_{Y_i}$ and $p^{(c)}_{Z_i}$ defined similarly.

Focusing on the causal effect of $\{Y\}$ on $\{X\}$, we now define the restricted conditional probability of $X_i$ as $p_{X_i}^{(r)}:\mcal{X}^{i-1} \times \mcal{Z}^{i-1} \rightarrow [0,1]$, where:

\begin{equation*}
p^{(r)}_{X_i}(x^{i-1},z^{i-1})
\triangleq \mathbb{P}(X_i = 1 \mid X^{i-1} = x^{i-1},Z^{i-1} = z^{i-1})
\end{equation*}

It is important to note that the restricted distribution is a function of the \emph{entire} past. This is a result of the fact that the joint-Markovicity of the complete collection of processes does not imply that a subset of the processes will be Markov. To see this, we first note that we can use the Markov property to derive the complete probability as:

\begin{equation*}
\begin{aligned}
\mathbb{P}&(X_i = 1 \mid X^{i-1} = x^{i-1},Y^{i-1} = y^{i-1},Z^{i-1} = z^{i-1}) \\
&= \mathbb{P}(X_i = 1 \mid X_{i-k}^{i-1} = x_{i-k}^{i-1},Y_{i-k}^{i-1} = y_{i-k}^{i-1},Z_{i-k}^{i-1} = z_{i-k}^{i-1}).
\end{aligned}
\end{equation*}

\noindent Now, if we attempt to derive the restricted probability of $X_i$ by marginalizing over possible outcomes of $Y^{i-1}$, we get (for compactness, the variables are omitted and only the values are shown):

\begin{equation*}
\begin{aligned}
&p_{X_i}^{(r)}(x^{i-1},z^{i-1}) = \sum_{y^{i-1}} \mathbb{P}(x_i,y^{i-1} \mid x^{i-1},z^{i-1}) \\
&= \sum_{y^{i-1}}
\mathbb{P}(x_i \mid x^{i-1},y^{i-1} ,z^{i-1})
\cdot \mathbb{P}(y^{i-1}\mid x^{i-1},z^{i-1}) \\
&= \sum_{y^{i-1}}
\mathbb{P}(x_i \mid x_{i-k}^{i-1},y_{i-k}^{i-1} ,z_{i-k}^{i-1})
\cdot \mathbb{P}(y^{i-1}\mid x^{i-1},z^{i-1}) \\
\end{aligned}
\end{equation*}

\begin{comment}
We begin by demonstrating the estimation of the proposed causal measure on a collection of binary $k^{th}$-order Markov processes $\{X\}$, $\{Y\}$, and $\{Z\}$. In particular, we define the complete history as $H^{(c)}_i =[x_{i-k}^{i-1},y_{i-k}^{i-1},z_{i-k}^{i-1}]^T \in \mathbb{R}^{3 \times k}$, and characterize the complete distribution by the following generalized linear models:


\begin{align}
\mathbb{P}(X_i = 1 | H^{(c)}_i) =
    \frac{e^{\mu^{(c)}_X + \langle \theta^{(c)}_X, H^{(c)}_i \rangle}}
    {1+e^{\mu^{(c)}_X + \langle \theta^{(c)}_X, H^{(c)}_i \rangle}} \\
\mathbb{P}(Y_i = 1 | H^{(c)}_i) =
    \frac{e^{\mu^{(c)}_Y + \langle \theta^{(c)}_Y, H^{(c)}_i \rangle}}
    {1+e^{\mu^{(c)}_Y + \langle \theta^{(c)}_Y, H^{(c)}_i \rangle}} \\
\mathbb{P}(Z_i = 1 | H^{(c)}_i) =
    \frac{e^{\mu^{(c)}_Z + \langle \theta^{(c)}_Z, H^{(c)}_i \rangle}}
    {1+e^{\mu^{(c)}_Z + \langle \theta^{(c)}_Z, H^{(c)}_i \rangle}} \\
\end{align}

\noindent where $\mu^{(c)}_* \in \mathbb{R}$ and $\theta^{(c)}_* \in \mathbb{R}^{3\times k}$ are model parameters and the inner product term may decomposed as:

\begin{equation}
\langle \theta^{(c)}_*, H^{(c)}_i \rangle =
    {\theta_{*X}^{(c)}}^T x_{i-k}^{i-1} +
    {\theta_{*Y}^{(c)}}^T y_{i-k}^{i-1} +
    {\theta_{*Z}^{(c)}}^T z_{i-k}^{i-1}
\end{equation}

\noindent with $\theta^{(c)}_{**} \in \mathbb{R}^k$. Focusing without loss of generality on the causal influence of $\{Y\}$ on $\{X\}$, \cyx, we define the restricted history as $H^{(r)}_i =[x_{i-k}^{i-1},z_{i-k}^{i-1}]^T \in \mathbb{R}^{2 \times k}$ and the restricted distribution:

\begin{equation}
\mathbb{P}(X_i = 1 | H^{(r)}_i) =
    \frac{e^{\mu^{(r)}_X + \langle \theta^{(r)}_X, H^{(r)}_i \rangle}}
    {1+e^{\mu^{(r)}_X + \langle \theta^{(r)}_X, H^{(r)}_i \rangle}}
\end{equation}

\noindent with $\mu^{(r)}_* \in \mathbb{R}$ and $\theta^{(r)}_* \in \mathbb{R}^{2\times k}$ and

\begin{equation}
\langle \theta^{(r)}_X, H^{(r)}_i \rangle = {\theta_{XX}^{(r)}}^T x_{i-k}^{i-1} + {\theta_{XZ}^{(r)}}^T z_{i-k}^{i-1}.
\end{equation}

It is clear that $\cyx(i) = 0$ for all $i=1,2,\dots$ if and only if
\end{comment}
\section{Discussion}

%%%%%%%%%%%%%%%%%%%%%%%%%%%%%%%%%%%%%%%%%%%%%%%%%%%%%%%%%%%%%%%%%%%%%%%%%%%%%%%

\ifCLASSOPTIONcaptionsoff
  \newpage
\fi

\end{document}


