\section{Discussion} \label{discussion}

We have presented a non-negative measure of local causal influence that captures the time-varying nature of causal relationships that is inherent to both stationary and non-stationary settings. Furthermore, we have shown that under mild assumptions, the finite sample performance of an estimator of the measure can be determined as a function of the worst-case regret of the sequential predictors used to implement the estimator. Finally, we have shown how even basic approaches to estimating the causal estimate in a non-stationary setting can capture the variations in causal influence that occur over small and large time scales.

It is important to note that the proposed causal measure does not solve the problem of estimating causal influence in time-varying settings, but rather it provides a perspective on causal influence that is naturally extended to any setting for which there are effective sequential prediction algorithms. By conditioning on the observed past, we avoid need to decide a window length (to approximate an expectation) when estimating time-varying DI and TE.

There are a number of directions for continued research in this area. First and foremost is the estimation of the measure on real data. This could begin to move past simply identifying the direction of information flow between real-word processes to identifying particular patterns for which the causal influence is greatest. This gives rise to a second direction of continued work, namely calculating the causal regret for specific estimators and carefully characterizing the circumstances for which the assumptions hold. Lastly, given that sequential prediction is not limited to discrete spaces, extending the causal measure for continuous settings will be a subject of continued work.